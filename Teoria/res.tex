
%%% Introduccion a la inteligencia artificial %%%

\section{Definición de IA}

La inteligencia artificial es la parte de las Ciencias de la Computación que
se ocupa del diseño de sistemas inteligentes, esto es, sistemas que exhiben
características que asociamos con la inteligencia en las conductas humanas.\\
Feigenbaum y Barr, '80s

Es estudio de cómo lograr que las computadoras realicen tareas que, por el
momento, los humanos hacen mejor.\\
E. Rich y Knight, 1991

La rama de las Ciencias de la Computación que se ocupa de la automatización
de la conducta inteligente.\\
Luger y Stubblefield, 1993

Es la Ciencia e Ingeniería de hacer máquinas inteligentes (especialmente
programas). Esto está relacionado a la tarea de usar computadoras para
entender la inteligencia humana, pero IA no tiene que limitarse a métodos
que son biológicamente observables.\\
J. McCarthy, 1998

Las definiciones se agrupan en:
\begin{tabular}{| c | c |}
\hline
Sistemas que piensan como humanos &
Sistemas que piensan racionalmente \\
\hline
Sistemas que actuan como humanos &
Sistemas que actuan racionalmente
\hline
\end{tabular}

Ramas de IA:
\begin{itemize}
    \item Búsqueda heurística
    \item Representación del conocimiento
    \item Inferencia
    \item Planificación
    \item Aprendizaje
    \item Lenguaje natural
    \item Visión
    \item Robótica
\end{itemize}

\textbf{Agente} es todo aquello que percibe su ambiente mediante sensores
y que responde o actúa mediante efectores. Un \textbf{agente inteligente}
debe hacer siempre lo correcto de acuerdo a sus percepciones. Es aquel que
emprende la mejor acción posible en una situación dada. La noción estándar
de un agente inteligente es: un sistema con las propiedades de autonomía,
habilidad social, reactividad y proactividad.

\textbf{Sistemas multi-agente}: Cada agente tiene información y capacidades
limitadas para resolver un problema. Puede no existir un control global del
sistema. Los datos están descentralizados y la computación es asíncrona.
Proveen más robustez, eficiencia y permiten la interoperatividad de sistemas
existentes.

\section{Representación de problemas y búsqueda sin información}
¿Cómo formular un problema de modo que permita la construcción de un proceso
para la búsqueda de soluciones?
\begin{itemize}
    \item Formulación de problemas mediante espacio de estados (representación
        formal)
    \item Aplicación de distintas técnicas de búsqueda (secuencia de operadores)
\end{itemize}
Una solución es una secuencia de acciones (un plan) que permite transformar el
estado inicial en un estado meta.\\
En la definición de operadores es útil tener en cuenta:
\begin{itemize}
    \item Suposiciones presentes en la descripción informal del problema que no
        están expresadas como tales (es útil representarlas).
    \item Nivel de generalidad de las reglas (recordar presentar casos
        especiales de utilidad).
    \item No incluir reglas inutiles.
    \item Pensar en un conjunto que resuelva más de un caso.
\end{itemize}
\textbf{Grafo de espacio de estados}: es una representación matemática de un
problema de búsqueda. Los nodos son configuraciones del mundo (abstracciones).
Los arcos representan sucesores (resultados de las acciones). El test de goal
es un conjunto de nodos meta (puede ser uno solo). En un grafo de búsqueda,
cada estado ocurre una sola vez. Raramente se construye todo el grafo en memoria
(es muy grande), pero es una buena idea.

Un árbol de búsqueda es un árbol de planes "what if" y sus salidas. El nodo de
inicio es el nodo raíz. Los hijos corresponden a sucesores. Los nodos representan
estados, pero corresponden a \textit{planes} para alcanzar esos estados. Para
la mayoría de los problemas no se construye todo el árbol.

Cada nodo en un árbol de búsqueda representa todo un camino en el grafo. Se
construyen ambos según demanda, tan pequeños como sea posible.

\textbf{Búsqueda}: Es el proceso de evaluar las distintas secuencias de acciones
(planes) para encontrar las que me lleven del estado inicial al estado meta. Es
muy importante en la solución de problemas de IA, si no existen técnicas más
directas.

Las estrategias de búsqueda deben ocasionar cambios, si no nunca alcanzará una
solución del problema. También deben ser sistemáticas, es necesario un cambio
global (en el curso de varios pasos), esto evita la reiteración de secuencias de
operadores poco apropiados. Y deben ser eficientes: completas, menor complejidad
posible (espacial y temporal), y una buena solución (si es óptima mejor).

\textbf{Enfoque de agentes}: Un agente formula una meta y debe esforzarse por
alcanzarla. Formulará el problema a encarar dependiendo de si conoce su estado
actual y si conoce el resultado de sus acciones, esto lo hace a través de
\textit{sensores}.

Agregamos a la definición de \textit{problema}: Prueba de meta (¿es el estado
actual una meta?) y costo de ruta (g), que es la suma de las acciones
individuales (operadores) que se emprenden al recorrerla.

El verdadero arte de la solución de problemas consiste en decidir qué es lo que
servirá para representar los estados y operadores, y qué no. Hay que realizar un
proceso de eliminación de los detalles que sean innecesarios -> Abstracción.

\textbf{Estrategias de búsqueda}
\begin{description}
    \item[Búsqueda sin información] El agente sólo puede diferenciar un nodo que
        es meta de uno que no lo es. No posee información respecto a cuántos pasos
        necesita dar, o a qué distancia está de la meta.
    \item[Búsqueda con información] El agente posee información sobre el problema
        para elegir operadores más convenientes.
\end{description}
Las estrategias de búsqueda sin información se diferencias por el orden en que se
expanden los nodos.

\textbf{Propiedades de los algoritmos de búsqueda}
\begin{itemize}
    \item Completitud: Garantiza encontrar una solución si existe alguna.
    \item Optimidad: Garantiza encontrar la solución óptima (menor costo).
    \item Complejidad espacial y temporal.
\end{itemize}
Nomenclatura adicional: $b$ es el factor de ramificación, $m$ es la profundidad
máxima.

\subsection{Búsqueda a lo ancho (Breadth-First Search)}
Estrategia: expandir los nodos menos profundos primero. Implementa una lista de
nodos como FIFO. Procesa todos los nodos arriba de la solución menos profunda.
Si la solución está a profundidad $s$, el tiempo de búsqueda es $O(b^s)$.
La complejidad espacial (nodos de la frontera) es aprox. $O(b^s)$.\\
El algoritmo es completo, $s$ debe ser finito si la solución existe, y es
óptimo, considerando todos los costos de operadores 1.

\subsection{Búsqueda de costo uniforme}
Expande siempre el nodo de menor costo. El costo de ruta asociado a cada nodo $n$
es $g(n)$, si $g(n) = profundidad(n)$, la búsqueda es preferente por amplitud (a lo
ancho). Garantiza obtener la solución más barata si el costo de ruta nunca disminuye
al avanzar ($g(sucesor(n)) \geq g(n)$).

El algoritmo es completo, y es óptimo siempre que el costo asociado al árbol de
búsqueda no sea decreciente en profundidad. La complejidad, espacial y temporal,
es $O(b^s)$, donde $s$ es la profundidad de la solución.

\subsection{Búsqueda preferente en profundidad (Deep-First Search)}
Se expande siempre uno de los nodos que se encuentra en el nivel más profundo.
Sólo cuando un nodo no tiene expansión se revierte la búsqueda y se expanden
nodos de niveles menos profundos. En este algoritmo sólo debe guardarse la ruta
y los nodos no expandidos, por lo cual no necesita tanta memoria como otros
algoritmos, BFS por ejemplo.

La complejidad espacial de DFS es $O(b.m)$, donde $m$ es la profundidad máxima
del árbol de búsqueda. Su complejidad temporal es $O(b^m)$. No es óptima, ya que
puede encontrar antes una solución que no es la mejor; ni es completa, puede
atascarse en bucles o caminos $\infty$. No es aconsejable utilizarla si el
árbol es de gran profundidad máxima.

\subsection{Búsqueda limitada por profundidad}
Se impone un límite máximo a la profundidad de la ruta, con lo cual se elimina
la posibilidad de atascamientos de DFS. Dado el límite $l$, su complejidad
espacial es $O(b.l)$, mientras su complejidad temporal es $O(b^l)$. Como DFS,
no es óptima, pero es completa si se elige el límite adecuado para el problema.
Si el límite es demasiado pequeño no puede garantizarse completitud, es
aconsejable donde se tenga idea de un límite razonable.

\subsection{Búsqueda por profundización iterativa}
Resuelve el problema que tiene BLP de elegir un límite adecuado. Propone probar
todos los límites, empezando de 0. Combina las ventajas de
\begin{itemize}
    \item DFS: Menor consumo de memoria
    \item BFS: Completa y óptima
\end{itemize}
Su complejidad espacial es $O(b.d)$, en tanto que la temporal es $O(b^d)$.
La desventaja es que los nodos de niveles altos se expanden varias veces. Es
aconsejable en espacios grandes donde se ignora la profundidad de la solución.

\subsection{Comparación de las estrategias}
\begin{tabular}{| c | c | c | c | c |}
\hline
Criterio & Tiempo & Espacio & Óptima? & Completa? \\
\hline
BFS & $b^d$ & $b^d$ & Sí & Sí \\
\hline
Costo uniforme & $b^d$ & $b^d$ & Sí & Sí \\
\hline
DFS & $b^m$ & $b.m$ & No & No \\
\hline
Limitada en profundidad & $b^l$ & $b.l$ & No & Cuando $l \geq d$ \\
\hline
Profundización iterativa & $b^d$ & $b.d$ & Sí & Sí \\
\hline
Bidireccional & $b^{\frac{d}{2}}$ & $b^{\frac{d}{2}}$ & Sí & Sí
\hline
\end{tabular}

\subsection{El problema de los estados repetidos}
En algunos problemas se llega a un estado de una sola forma, y por ende
es imposible repetir. Cuando los operadores son reversibles, entonces
pueden obtenerse árboles infinitos (misioneros y caníbales, rutas). Evitar
las repeticiones disminuye sensiblemente el costo de la búsqueda.

Formas de evitar los estados repetidos:
\begin{enumerate}
    \item No regresar al estado del que acaba de llegar, $sucesor(n) \neq n$.
    \item No generar rutas que tengan ciclos, $sucesor(n) \neq \{n\}
        \cup ancestros(n)$.
    \item No generar ningún estado que se haya generado alguna vez.
\end{enumerate}
Hay un compromiso entre el costo de almacenar y verificar, y el costo de 
la búsqueda adicional a realizar.



\section{Búsqueda con información}
Para reducir la extensión de la búsqueda desinformada debemos incorporarle
conocimiento. Al contar con información específica sobre un espacio de estados,
se evitan emprender búsquedas a ciegas. Con métodos generales más heurísticas
de propósito especial, se obtiene mayor eficiencia.

Se utiliza una \textbf{función heurística} para representar lo deseable que es
la expansión de un nodo. Una función heurística tiene tipo $f : Estados \to Num$
El problema de búsqueda se puede considerar como la minimización de una función.
Si $f$ está bien diseñada, guía la búsqueda eficientemente. Una $f$ ideal
establece el camino a la meta. \textit{Problema: Elegir buenas heurísticas}.

\subsection{Búsqueda primero el mejor (Best First)}
Los nodos se ordenan de tal manera que se expande el nodo de mejor valor de la
función heurística $f$ (mínimo, máximo), esta función puede incorporar
conocimiento del dominio. Si $f(n) = g(n)$, donde $g(n)$ es el costo de ruta,
la búsqueda se transforma en una de costo uniforme.

\subsection{Búsqueda de costo mínimo (Greedy)}
Implementa búsqueda primero el mejor, buscando el mínimo de una función $h(n)$
que representa el costo estimado para lograr llegar del estado $n$ a un estado
meta. Si $m$ es un nodo meta, $h(m) = 0$; y si $l$ es una hoja del árbol desde
la cual no se puede llegar a un nodo meta, $h(l) = \infty$. Esta búsqueda no 
es óptima, ni completa (es susceptible a pasos en falso). Su complejidad
espacial y temporal es $O(b^m)$, ya que mantiene todos los nodos en memoria,
aunque si $h$





